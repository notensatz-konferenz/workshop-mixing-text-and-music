% Options for packages loaded elsewhere
\PassOptionsToPackage{unicode}{hyperref}
\PassOptionsToPackage{hyphens}{url}
%
\documentclass[
]{mecExercise}
\usepackage{lmodern}
\usepackage{amssymb,amsmath}
\usepackage{ifxetex,ifluatex}
\ifnum 0\ifxetex 1\fi\ifluatex 1\fi=0 % if pdftex
  \usepackage[T1]{fontenc}
  \usepackage[utf8]{inputenc}
  \usepackage{textcomp} % provide euro and other symbols
\else % if luatex or xetex
  \usepackage{unicode-math}
  \defaultfontfeatures{Scale=MatchLowercase}
  \defaultfontfeatures[\rmfamily]{Ligatures=TeX,Scale=1}
\fi
% Use upquote if available, for straight quotes in verbatim environments
\IfFileExists{upquote.sty}{\usepackage{upquote}}{}
\IfFileExists{microtype.sty}{% use microtype if available
  \usepackage[]{microtype}
  \UseMicrotypeSet[protrusion]{basicmath} % disable protrusion for tt fonts
}{}
\makeatletter
\@ifundefined{KOMAClassName}{% if non-KOMA class
  \IfFileExists{parskip.sty}{%
    \usepackage{parskip}
  }{% else
    \setlength{\parindent}{0pt}
    \setlength{\parskip}{6pt plus 2pt minus 1pt}}
}{% if KOMA class
  \KOMAoptions{parskip=half}}
\makeatother
\usepackage{xcolor}
\IfFileExists{xurl.sty}{\usepackage{xurl}}{} % add URL line breaks if available
\IfFileExists{bookmark.sty}{\usepackage{bookmark}}{\usepackage{hyperref}}
\hypersetup{
  hidelinks,
  pdfcreator={LaTeX via pandoc}}
\urlstyle{same} % disable monospaced font for URLs
\usepackage{color}
\usepackage{fancyvrb}
\newcommand{\VerbBar}{|}
\newcommand{\VERB}{\Verb[commandchars=\\\{\}]}
\DefineVerbatimEnvironment{Highlighting}{Verbatim}{commandchars=\\\{\}}
% Add ',fontsize=\small' for more characters per line
\newenvironment{Shaded}{}{}
\newcommand{\AlertTok}[1]{\textcolor[rgb]{1.00,0.00,0.00}{\textbf{#1}}}
\newcommand{\AnnotationTok}[1]{\textcolor[rgb]{0.38,0.63,0.69}{\textbf{\textit{#1}}}}
\newcommand{\AttributeTok}[1]{\textcolor[rgb]{0.49,0.56,0.16}{#1}}
\newcommand{\BaseNTok}[1]{\textcolor[rgb]{0.25,0.63,0.44}{#1}}
\newcommand{\BuiltInTok}[1]{#1}
\newcommand{\CharTok}[1]{\textcolor[rgb]{0.25,0.44,0.63}{#1}}
\newcommand{\CommentTok}[1]{\textcolor[rgb]{0.38,0.63,0.69}{\textit{#1}}}
\newcommand{\CommentVarTok}[1]{\textcolor[rgb]{0.38,0.63,0.69}{\textbf{\textit{#1}}}}
\newcommand{\ConstantTok}[1]{\textcolor[rgb]{0.53,0.00,0.00}{#1}}
\newcommand{\ControlFlowTok}[1]{\textcolor[rgb]{0.00,0.44,0.13}{\textbf{#1}}}
\newcommand{\DataTypeTok}[1]{\textcolor[rgb]{0.56,0.13,0.00}{#1}}
\newcommand{\DecValTok}[1]{\textcolor[rgb]{0.25,0.63,0.44}{#1}}
\newcommand{\DocumentationTok}[1]{\textcolor[rgb]{0.73,0.13,0.13}{\textit{#1}}}
\newcommand{\ErrorTok}[1]{\textcolor[rgb]{1.00,0.00,0.00}{\textbf{#1}}}
\newcommand{\ExtensionTok}[1]{#1}
\newcommand{\FloatTok}[1]{\textcolor[rgb]{0.25,0.63,0.44}{#1}}
\newcommand{\FunctionTok}[1]{\textcolor[rgb]{0.02,0.16,0.49}{#1}}
\newcommand{\ImportTok}[1]{#1}
\newcommand{\InformationTok}[1]{\textcolor[rgb]{0.38,0.63,0.69}{\textbf{\textit{#1}}}}
\newcommand{\KeywordTok}[1]{\textcolor[rgb]{0.00,0.44,0.13}{\textbf{#1}}}
\newcommand{\NormalTok}[1]{#1}
\newcommand{\OperatorTok}[1]{\textcolor[rgb]{0.40,0.40,0.40}{#1}}
\newcommand{\OtherTok}[1]{\textcolor[rgb]{0.00,0.44,0.13}{#1}}
\newcommand{\PreprocessorTok}[1]{\textcolor[rgb]{0.74,0.48,0.00}{#1}}
\newcommand{\RegionMarkerTok}[1]{#1}
\newcommand{\SpecialCharTok}[1]{\textcolor[rgb]{0.25,0.44,0.63}{#1}}
\newcommand{\SpecialStringTok}[1]{\textcolor[rgb]{0.73,0.40,0.53}{#1}}
\newcommand{\StringTok}[1]{\textcolor[rgb]{0.25,0.44,0.63}{#1}}
\newcommand{\VariableTok}[1]{\textcolor[rgb]{0.10,0.09,0.49}{#1}}
\newcommand{\VerbatimStringTok}[1]{\textcolor[rgb]{0.25,0.44,0.63}{#1}}
\newcommand{\WarningTok}[1]{\textcolor[rgb]{0.38,0.63,0.69}{\textbf{\textit{#1}}}}
\setlength{\emergencystretch}{3em} % prevent overfull lines
\providecommand{\tightlist}{%
  \setlength{\itemsep}{0pt}\setlength{\parskip}{0pt}}
\setcounter{secnumdepth}{-\maxdimen} % remove section numbering

\author{}
\date{}

\begin{document}

\hypertarget{markdown-with-class}{%
\section{\texorpdfstring{Markdown With
\LaTeX~Class}{Markdown With ~Class}}\label{markdown-with-class}}

This is also a markdown file, but one that is more closely tied to
\LaTeX, in so far as it references a document class. This way we can
reference anything defined in the \LaTeX~class, but it will only work
with the Markdown-to-PDF toolchain. But assuming \emph{some}
preparations this is a way to provide a very natural environment to
easily perform recurring tasks such as teaching or test sheets.

\lilypond{ \clef bass } \}

\begin{itemize}
\tightlist
\item
  sdlkfj
\item
  skdf just

  \begin{itemize}
  \tightlist
  \item
    sdfkjls
  \end{itemize}
\end{itemize}

The following exercise has been entered in the text document by a single
invocation of

\begin{Shaded}
\begin{Highlighting}[]
\FunctionTok{\textbackslash{}exercise}\NormalTok{\{Bass Figures\}\{01{-}bassline\}}
\end{Highlighting}
\end{Shaded}

which does all the formatting, introduces a music example with certain
styles and takes care of the numbering. The music is retrieved from a
LilyPond file that is referenced by the second argument
\texttt{01-bassline}.

\exercise{Bass Figures}{01-bassline}

The \LaTeX~command also accepts an optional argument to be typeset
differently:

\begin{Shaded}
\begin{Highlighting}[]
\FunctionTok{\textbackslash{}exercise}\NormalTok{[Please complete the bass line first]\{Bach Chorale\}\{02{-}chorale\}}
\end{Highlighting}
\end{Shaded}

\exercise[Please complete the bass line first]{Bach Chorale}{02-chorale}

\pagebreak

\exercise{Frames}{04-frames}

\exercise{Arrows}{05-arrows}

Another example, just to show how e.g.~harmonic analysis can be
typeset/engraved with LilyPond:

\exercise{Analysis}{03-alle-meine-entchen}

For reference: This is how the last exercise has been coded:

\begin{Shaded}
\begin{Highlighting}[]
\FunctionTok{\textbackslash{}relative} \KeywordTok{\{}
\NormalTok{  c\textquotesingle{}}\DataTypeTok{4}\NormalTok{ d e f}
\NormalTok{  g}\DataTypeTok{2}\NormalTok{ g}
\NormalTok{  a}\DataTypeTok{4}\NormalTok{ a a a}
\NormalTok{  g}\DataTypeTok{1*1/2} \KeywordTok{\textbackslash{}once} \FunctionTok{\textbackslash{}hideNotes}\NormalTok{ g}
\NormalTok{  a}\DataTypeTok{4}\NormalTok{ a a a}
\NormalTok{  g}\DataTypeTok{1}
\NormalTok{  f}\DataTypeTok{4}\NormalTok{ f f f}
\NormalTok{  e}\DataTypeTok{2}\NormalTok{ e}\DataTypeTok{2*1/2} \KeywordTok{\textbackslash{}once} \FunctionTok{\textbackslash{}hideNotes}\NormalTok{ e}
\NormalTok{  g}\DataTypeTok{4}\NormalTok{ g g g}
\NormalTok{  c,}\DataTypeTok{1}
  \FunctionTok{\textbackslash{}bar}\NormalTok{ "}\StringTok{|."}
\KeywordTok{\}}
\NormalTok{\textbackslash{}addlyrics}\KeywordTok{ \{}
  \FunctionTok{\textbackslash{}lyricsToFunctions}
\NormalTok{  "}\StringTok{T"}\NormalTok{ "}\StringTok{/D\_3{-}7"}\NormalTok{ "}\StringTok{(D\_7)"}\NormalTok{ "}\StringTok{S\_3"}
\NormalTok{  "}\StringTok{T\_5"}\NormalTok{ "}\StringTok{\_3"}
\NormalTok{  "}\StringTok{Sp\_3"} \KeywordTok{\_}\NormalTok{ "}\StringTok{DD\_3{-}7"} \KeywordTok{\_}
\NormalTok{  "}\StringTok{D{-}4{-}6"}\NormalTok{ "}\StringTok{{-}3{-}5"}
\NormalTok{  "}\StringTok{DD{-}v\^{}5"} \KeywordTok{\_} \KeywordTok{\_} \KeywordTok{\_}
\NormalTok{  "}\StringTok{tP\_3"}
\NormalTok{  "}\StringTok{(D{-}v\_5>"}\NormalTok{ "}\StringTok{s\_5"}\NormalTok{ "}\StringTok{/D{-}5>{-}7{-}9>"}\NormalTok{ "}\StringTok{D{-}7{-}9>)"}
\NormalTok{  "}\StringTok{Tp"}\NormalTok{ "}\StringTok{(D\_5{-}7)"}\NormalTok{ "}\StringTok{[S]"}
\NormalTok{  "}\StringTok{dG\_3"} \KeywordTok{\_}\NormalTok{ "}\StringTok{D{-}6"}\NormalTok{ "}\StringTok{{-}5{-}7"}
\NormalTok{  "}\StringTok{T"}
\KeywordTok{\}}
\end{Highlighting}
\end{Shaded}

\listofexercises

\end{document}
